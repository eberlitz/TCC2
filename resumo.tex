%=======================================================================
% Resumo em língua estrangeira (sim, é aqui mesmo).
%
% O idioma usado aqui deve necessariamente aparecer nos parâmetros do
% \documentclass, no início do documento.
%=======================================================================
\begin{otherlanguage}{brazilian}
\othertitle{Modelo semântico para similaridade de palavras}
\begin{abstract}
A capacidade de identificar a similaridade semântica entre palavras tem sido objeto de pesquisa nos últimos anos, pois está relacionada a uma série de atividades da área de processamento de linguagem natural, como recuperação de informação, sumarização de texto, categorização e geração, tradução automática e outros. A maioria dos sistemas de resposta a perguntas e extração de informações usa o WordNet para procurar sinônimos em seu mecanismo de busca. No entanto, a expansão de termos usando o WordNet tem vários problemas. Eles são de construção manual, demorados e caros, e por esse motivo, nem todos os links estarão presentes e sua qualidade varia de idioma para idioma, assim como não está disponível para todos os idiomas. Abordagens baseadas em distribuição, como a \textit{word embedding}, foram usadas para cobrir itens fora do vocabulário no WordNet. Assim, com a possibilidade de acesso a \textit{word embeddings} pré-treinados incluindo na língua portuguesa e a necessidade de melhorar a forma de expandir os termos relacionados aos sistemas de consulta para bases ontológicas utilizadas por sistemas de perguntas e respostas e recuperação de informação, o presente trabalho visa melhorar a precisão e o recall desses termos relacionados por meio do uso de word embeddings. Além disso, propomos adaptar os estudos existentes sobre o contexto sintático no processo de formação de word embeddings para um corpus do português brasileiro, para verificar se os resultados serão semelhantes ou não. A avaliação será composta por análises qualitativas e também quantitativas em dois conjuntos de dados comuns.
\palavraschave{Similaridade de palavras. WordNet. Word embedding. Linguística computacional. Processamento de Linguagem Natural.}
\end{abstract}
\end{otherlanguage}