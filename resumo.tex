%=======================================================================
% Resumo em língua estrangeira (sim, é aqui mesmo).
%
% O idioma usado aqui deve necessariamente aparecer nos parâmetros do
% \documentclass, no início do documento.
%=======================================================================
\begin{otherlanguage}{brazilian}
\othertitle{Modelo semântico para similaridade de palavras}
\begin{abstract}
A capacidade de identificar a similaridade semântica entre palavras tem sido objeto de pesquisa nos últimos anos, pois está relacionada a uma série de atividades da área de processamento de linguagem natural, como recuperação de informação, sumarização de texto, categorização e geração, tradução automática e outros. A maioria dos sistemas de resposta a perguntas e extração de informações usa o WordNet para procurar sinônimos em seu mecanismo de busca. No entanto, a expansão de termos usando o WordNet tem vários problemas. Eles são de construção manual, demorados e caros, e por esse motivo, nem todos os links estarão presentes e sua qualidade varia de idioma para idioma, assim como não está disponível para todos os idiomas. Abordagens baseadas em distribuição, como a \textit{word embedding}, foram usadas para cobrir itens fora do vocabulário no WordNet. Assim, com a possibilidade de acesso a diferentes \textit{word embeddings models} e a necessidade de melhorar a forma de expandir os termos relacionados aos sistemas de consulta para bases ontológicas utilizadas por sistemas de perguntas e respostas e recuperação de informação, o presente trabalho explora as técnicas existentes para identificação de similaridade entre palavras, usando a abordagem distribuicional chamada word embeddings, adaptandando trabalhos existentes para o português brasileiro. Também é realizado experiementos com outras tecnicas que são basesadas em bases lexicas como WordNet, aonde uma avaliação qualitativa é realizada de todas as técnicas sobre um \textit{dataset} comun PT65. Provando que word embeddings podem de fato cobrir as palavras faltantes e tem um resultado ligeiramente melhor quando comparado com o WordNet. Também é realizado uma adaptação dos estudos de \citetexto{Levy2014} no que tange a adição de contexto sintático no processo de trainamento do word embeddings a partir e um corpus português brasileiro, aonde obtivemos resultados similares atraves de uma avaliação qualitativa, porem para a atividade de identificar palavras similares utilizando o dataset PT65 os resultados foram piores em considersão aos outros modelos.
\palavraschave{Similaridade de palavras. WordNet. Word embedding. Linguística computacional. Processamento de Linguagem Natural.}
\end{abstract}
\end{otherlanguage}