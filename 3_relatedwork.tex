% Foram apresentados trabalhos relacionados ao tema abordado?
% Os trabalhos relacionados são relevantes para o tema investigado?
% Os trabalhos relacionados foram comparados com a pesquisa realizada?
% Os trabalhos relacionados são recentes?  Últimos 5 anos?
% Os trabalhos relacionados fizeram parte da discussão dos resultados da pesquisa realizada?



% Qual o problema abordado? hipotese?
% O que foi feito? Solução?
% Como foi realizada a avaliação?
% Comparar com o o meu trabalho - porque é relevante ao meu trabalho?


% G
\section{Related work}\label{chap:relatedwork}

In this section, will be presented works encountered while doing the bibliographic research. To find the state-of-the-art regarding word similarity, a search with Google Scholar\footnote{\url{https://scholar.google.com.br/}} and Semantic Scholar\footnote{\url{https://www.semanticscholar.org/}} was used. The terms used to find the related work in the field was "word similarity", "semantic similarity", "synonym", "Synonym extraction", "semantic embedding" and "morphological embedding". Also, some search through the Association for Computational Linguistics website\footnote{\url{https://aclweb.org/aclwiki/Similarity_(State_of_the_art)}} revealed some events regarding Semantic Textual Similarity, like the SemEval which were used to search for articles.

% TODO?: \subsection{trabalho denis}

\subsection{Comparing Semantic Relatedness between Word Pairs in Portuguese Using Wikipedia}

In this paper, \citetexto{GranadaSV14} presents a new dataset for evaluating Distributional Similarity Models in Portuguese. For this, they translated the word pairs from the well-know baseline for semantic relatedness evaluation in English called RG65 created by \citetexto{Rubenstein1965ContextualCO}. The original dataset contains judgements from 51 human subjects for 65 word pairs. To generate the PT65 they translated all the word-pairs and evaluated them with 50 human subjects. They compared the human scores with previous works and also performed a qualitative evaluation using Latent semantic analysis (LSA) models generated from Wikipedia articles. The correlation scores obtained were closer to the scores achieved by other works that targeted another language. With the experiement they observed that the semantic similarity can be transfered across languages, but for Portuguese, a manual evaluation had better results. 

With this, we intend to use the PT65 as a gold-standard for evaluating semantic similarity and relatedness between words with word embeddings models and the WordNet.

 

% Qual o problema abordado? hipotese?
% O que foi feito? Solução?
% Como foi realizada a avaliação?
% Comparar com o o meu trabalho - porque é relevante ao meu trabalho?


% G
\subsection{Dependency-Based Word Embeddings}

In this work \citetexto{Levy2014} presents a generalized skip-gram model with negative sampling introduced by \citetexto{Mikolov2013DistributedRO}, from a linear context of bag-of-words to arbitrary word contexts, specifically syntactic contexts. An interesting fact of this approach in comparison with the original work is that the concept of induced similarity represents a nature of \textit{cohyponym}. They also describe a way of performing an analysis of the representation learned in the vector space by exploring the contexts of specific words or a group of words.
They used the English Wikipedia as a corpus to train the embeddings. This corpus was tagged with parts-of-speech (POS) using the Stanford tagger. 

For the evaluation, they manually inspected the five most similar words to a hand-picked set of words. One remarkable example is the word "Hogwarts" that in the BoW model the most similar words are from the respective domain of Harry Potter and in the developed model it was a list of famous schools, that is, was able to capture the semantic type of the word.
The model was evaluated against the WordSim353 dataset from \citetexto{Finkelstein:2001:PSC:371920.372094}, which is a dataset regarding word similarity versus relatedness. They draw a precision-recall curve that describes the embeddings affinity, proving that the results obtained by the developed model were slightly better than the BoW model.



% Qual o problema abordado? hipotese?
% O que foi feito? Solução?
% Como foi realizada a avaliação?
% Comparar com o o meu trabalho - porque é relevante ao meu trabalho?

\subsection{Portuguese Word Embeddings: Evaluating on Word Analogies and Natural
Language Tasks}

\citetexto{Hartmann2017} present in this paper, an evaluation of different word embedding models trained on a large Portuguese corpus (Brazilian and European variants together) on syntactic and semantic analogies, POS tagging and sentence semantic similarity tasks.

They collected a large corpus from various sources, either in Brazilian or European Portuguese. With that they applied some preprocessing (Tokenization and normalization) in order to reduce the vocabulary size. Using the corpus as input, they trained some word embedding models using four different algorithms (Word2Vec, Wang2Vec, FastText and GloVe) with varying dimensions (50, 100, 300, 600 and 1000).

For the evaluation, first, they use the syntactic and semantic analogies provided by \citetexto{Rodrigues2016LXDSemVectorsDS}. Where the FastText model performed better for syntactic analogies while for semantic analogies GloVe had the best performance. Also, all CBOW models, except Wang2Vec, had very low results in semantic analogies.

For the POS tagging task evaluation, the Wang2Vec had the best results, and as seen, higher dimensions had better performance. The worst models in this task was GloVe and FastText. 

For the sentence semantic similarity task evaluation, they used the ASSIN dataset. With this they had Word2Vec CBOW model with 1000 dimensions as the best one for European Portuguese. And for Brazilian Portuguese the Wang2Vec Skip-Gram model with 1000 dimensions had the best scores.
In the end, they suggest that word analogies are not very suitable for evaluating word embeddings and task specific is probably a better approach.

We used they pre-trained word embedding models for comparison while evaluating our models under our specific task - word similarity on PT65.








% G
\subsection{ELMo and BERT}

\citetexto{Peters:2018} presents in this work, a general approach for learning context-dependent representations from bidirectional language models (biLMs). They called it Embeddings from Language Models (ELMo), and we can image it as a new kind of word embedding, that, instead of learning a word as a vector representation it has the intent to catch the context of a word as a vector representation, meaning that, it learns embeddings with the different nuances of a single word. Models like GloVe, Word2Vec, Wang2Vec, and FastText would generalize all the different nuances of a single word in a single word vector having the same representation. With the release of ELMo, it brought near state-of-the-art results in many downstream NLP tasks,  including question answering, textual entailment, and sentiment analysis.

ELMo induced the current state-of-the-art technique called BERT, which stands for Bidirectional Encoder Representations from Transformers.  
BERT, a work by \citetexto{devlin2018bert}, is a method of pre-training language representations. It outperforms previous methods because it is the first unsupervised, deeply bidirectional system for pre-training NLP. It uses attention transformers instead of bidirectional RNNs to encode context.

As these works represent the state-of-the-art evolution from the first word embeddings and allow pre-trained models to be used for general purpose NLP tasks, we intend to explore how these language models behave for word-level tasks such as word similarity. 


\subsection{Distributional and Knowledge-Based Approaches for Computing Portuguese Word Similarity}

In this work \citetexto{gonccalo2018distributional}, presents several approaches for computing word similarity in Portuguese where they make an extensive comparison between these different approaches, which includes word embeddings and lexical knowledge bases. The comparison aims, for each approach, the effectiveness in this particular task as well as the quality and coverage of words. For the evaluation, they do a qualitative analysis of the results based on four datasets that have been recently adapted to Portuguese, PT65, SimLex-999, WordSim-353, and RareWords.
The results of this work indicate that distributional models capture relatedness better than lexical knowledge bases which seems to be better suited genuine similarity.


\subsection{Discussion}

Based on these works, we evaluated the WordNet against the word embedding approach regarding the identification of word similarity using several word embedding algorithms. 
We used the pre-trained word embedding models presented by \citetexto{Hartmann2017} for comparison while evaluating our models under our specific task of word similarity identification.
Moreover, we adapt the work of \citetexto{Levy2014} to generate a word embedding with syntactic contexts from a Brazilian Portuguese corpus, to check if the results were, similar or not regarding other models, but instead of using the WordSim353 which is for English, we used another dataset that can be considered a gold standard for our target language.
Because of the results presented by \citetexto{GranadaSV14}, we ended up using the PT65 as a gold-standard for evaluating semantic similarity and relatedness between words with word embedding models and the WordNet. 

Regarding the work from \citetexto{gonccalo2018distributional}, they did not evaluate different dimension sizes for the word embedding models, they also did not compare results with the DEPS word embeddings that include syntactic knowledge, so in our work we intend to do this evaluation and also compare results with our own generated word embeddings, instead of examining only the NILC available ones.


Regarding the works from \citetexto{Peters:2018} and \citetexto{devlin2018bert}, as they represent the state-of-the-art evolution from the first word embedding models and allow pre-trained models to be used for general purpose NLP tasks, we intend to explore how these language models behave for word-level tasks such as word similarity.

% \subsection{A study on similarity and relatedness using distributional and WordNet-based approaches}

In this paper, \citetexto{Agirre2009} compares the two main categories of techniques used to measure semantic similarity. Using graph-based algorithms to Word-Net and distributional similarities collected from a 1.6 Terabyte Web corpus. A joint of the two techniques are also explored.

For the graph-based algorithm they represent WordNet version 3.0 as a graph, where the relations among synsets are undirected edges, and for this graph, the PageRank is computed for each of the words in the corpus producing a probability distribution over synsets. Then, this is encoded as vectors by computing the cosine between them. In this word two WordNet versions were used, the WordNet 3.0 and the Multilingual Central Repository (MCR) aiming to link words between multiple WordNet languages. For cross-linguality, they exchange each non-English word in the dataset with its 5 best translations into English and then create the vector with the calculated similarities.

For the distributional approach of calculating similarities between words the explore the use of a vector space model using three variations as bag-of-words, context-window and syntactic-dependency over a corpus of four billion documents crawled from the web in August 2008.

They evaluate all the approaches over two standard datasets (RG65 and WordSim353) and also test a combination of both approaches (WordNet and Distributional) by training an SVM classifier to select the best result of the tree distributional variations for each pair. Thus, achieving state-of-the-art distributional and WordNet-based similarity measures over this datasets.

% \subsection{Morphological Word Embeddings}

% % TODO: Falta o como foi feito técnicamente
% % TODO: breve descrição do porque é relevante para o meu trabalho

% In this work, \citetexto{Cotterell2015MorphologicalW} propose a new model, Morph-LBL, for the semi-supervised induction of morphologically guided embedding. The raw text was annotated with morphological data with the intent to create word embeddings that preserve the morphological relations of the words. The motivation for doing this is the hypothesis that languages with a high morpheme per word ratio would have improved results if we take into account the morphological information of the words.

% They extend the log-bilinear model (LBL) by training with a corpus annotated with morphological tags. A very interesting point is that only a part of the corpus was annotated with the tags, only to initially guide the embeddings with the intention that they maintain their morphological characteristics during the rest of the training.
% Qualitative evaluation was performed by attempting to determine if a word close in the vector space model is also morphological close to another and in fact, they were.
% They also introduced a new metric for quantitative evaluation of the model, named MorphoDist, that they used to compare with other models, and the Morph-LBL surpassed the original Skip-Gram model and Log-Bilinear Model.

% \subsection{A Minimally Supervised Approach for Synonym Extraction with Word Embeddings}

% % TODO: Falta o como foi feito técnicamente
% % TODO: breve descrição do porque é relevante para o meu trabalho

% In this work, \citetexto{Leeuwenberga2016} investigates the use of word embeddings for automatic extraction of synonyms from a corpus. Their initial motivation came from machine translation evaluation where hypothesis translations are automatically compared with reference translations using a system that do this kind of evaluation named Meteor. Meteor is composed of four modules and one of them is synonym matching that currently uses WordNet for such a task. One problem with WordNet is that it is not available to all languages. So, the idea here is to use Word Embeddings a synonym matcher and in that case, it could be available to multiple languages as the training of word embeddings is unsupervised.
% They trained the word embeddings using three different approaches, CBoW, SG, and GloVe over English and German.  Also, they used a part-of-speech (POS) tagger to improve the synonym extraction. For evaluation, synonyms were obtained from WordNet 3.0 for English and GermaNet 10.0 for German.
% They excluded the results for the GloVe vectors, as they showed lower precision than SG and CBOW, and they did not use them in further experiments.
% From these experiments, they conclude that POS tags can help to slightly improve synonym extraction.

% ------------------------------------------------------------------
% \subsection{A Minimally Supervised Approach for Synonym Extraction with Word Embeddings}

% They use word embeddings aiming to be extensible to various languages. 

% They came up with a new similarity metric, relative cosine similarity, and show that this metric improves the extraction of synonyms from raw text. They also employ the extracted synonyms in the synonymy module of Meteor and use human evaluation to judge the quality of synonyms extracted.

% . Approaches like Meteor (Denkowski and Lavie, 2014; Banerjee and Lavie, 2005) computes an alignment between the hypothesis and reference to
% determine to what extent they convey the same meaning. Finding possible matches is done by means of four modules. One of them is synonym matching, that uses a synonym database resource to match words which may not be string identical. The module uses synonyms from the lexical database WordNet (Miller, 1995) and for this reason, not available for all languages. 
% Manual construction of lexical resources such as WordNet is time-consuming and expensive and needs to be done for each different language.

% Different experiments over the use of word embedding were carried out. First, they analyze the effect of contextual window size, the number of dimensions, and the type of word vectors on the precision of extraction, for English and German. Secondly, they look closely at the word categories that are (cosine) similar in the vector space. Then, they look at cosine similarity and introduce relative cosine similarity. Lastly, they examine the overlap of the most similar words in different vector spaces.

% They trained the word embeddings using three different approaches, CBoW, SG, and Global Vectors (GloVe) (Pennington et al., 2014) using different parameters with a 150 million word subset of the NewsCrawl corpus for English and German. Lowercasing, tokenization, and digit conflation were applied as preprocessing for both languages. Also, they used a part-of-speech (POS) tagger to improve the synonym extraction. For evaluation, synonyms were obtained from WordNet 3.0 for English and GermaNet 10.0 for German.

% They excluded the results for the GloVe vectors, as they showed lower precision than SG and CBOW, and they did not use them in further experiments. The general trends of the GloVe vectors were that they had higher precision for larger window sizes. The vectors with highest precision of 0.067 for English were of dimension 300, with a window size of 32. For German, the highest precision was 0.055, and the vectors were of dimension 1200, with a window size of 32 as well.

% For evaluation, they use the synonyms from WordNet 3.0 for English, and GermaNet 10.0 for German. In both WordNet and GermaNet words carry a corresponding part-of-speech. InWordNet these are nouns, verbs, adjectives, and adverbs. In GermaNet, synonyms are given for nouns, verbs, and adjectives. Because a given word’s part of speech is unknown here, we consider the synonyms of each word to be those of all the parts of speech it can potentially have inWordNet or GermaNet.
% 3.2.

% They develop a different measure to calculate similarity relative to the top n most similar words between word wi and wj as:

% \begin{equation}
%   rcs_n(w_i,w_j) = \frac{cosine_similarity(w_i,w_j)}{
%   \sum_{w_c\epsilon TOP_n} cosine_similarity(w_i,w_c)
%   }
% \end{equation}

% In order to separate some word senses we preprocessed both the English and German corpora from the previous chapter with the Stanford POS tagger (Toutanova et al., 2003), using the fastest tag-models. To compare using the simplified POS tags with the previous approaches we also calculated precision, recall and f-measure on Stest. Compared to the baseline of looking only at the most-similar word,we found that recall in English increased from 3\% to 4\%, precision did not change (11\%), and f-measure from 5\% to 6\%. Notably, German precision increased with 8\% to 12\%, recall from 5\% to 7\%, and f-measure from 6\% to 9\%. From these experiments we conclude that POS tags can help to improve synonym extraction in three ways. Firstly, they can separate some of the word senses, however this effect is minor. Secondly, they can filter words that are not grammatically similar enough, such as plurals. And thirdly, they can exclude synonyms in categories for which there no, or very few, synonyms, such as names.

% \subsection{UMBC EBIQUITY-CORE: Semantic Textual Similarity Systems}

% In this paper, \citetexto{Han2013} presents a hybrid word similarity model that was used in semantic text similarity systems developep for the *SEM 2013 STS shared task were they achieved first place of the 89 submitted runs.

% Their model was originally developed for the Graph of Relations project which maps informal queries with English words and phrases for an RDF linked data collection into a SPARQL query. The model combines LSA word similarity and WordNet knowledge.


