%=======================================================================
% Resumo em Português.
%
% A recomendação é para 150 a 500 palavras.
%=======================================================================
\begin{abstract}
The ability to identify the semantic similarity between words has been a subject of research very explored in the last years, because it is an important support to a series of activities of the area of natural language processing like information retrieval, text summarization, categorization and generation, database schema matching, question answering, machine translation, and others. For example, most of question answering and information extraction systems use WordNet to search for synonyms in their search engine. However, the expansion of terms using WordNet has several problems. They are of manual construction, time-consuming and expensive, and for this reason, not all links will be present and their quality varies from language to language, as also, it is not available for all languages. Distributional-based approaches like word embeddings have successfully been used to cover out-of-vocabulary items in WordNet. Thus, with the possibility of access to different word embeddings models and the need to improve the way of expanding related terms of query systems to ontological bases used by systems of questions answering and information retrieval, the present work explores the existing techniques regarding word similarity, using a distributional approach and adapting existing works to Brazilian Portuguese. We also experiment with other techniques that are solely based on a lexical database such as WordNet, and we do a qualitative evaluation of all the different techniques over a common dataset called PT65, proving that word embeddings can cover words out of vocabulary and have better results in comparison with WordNet. We also adapted the studies regarding the addition of syntactic context in the training process of word embeddings to a Brazilian Portuguese corpus, finding similar results through a qualitative evaluation.
\palavraschave{Word similarity. WordNet. Word embedding. Computational linguistics. Natural Language Processing.}
\end{abstract}

%=======================================================================
% Resumo em língua estrangeira (sim, é aqui mesmo).
%
% O idioma usado aqui deve necessariamente aparecer nos parâmetros do
% \documentclass, no início do documento.
%=======================================================================
\begin{otherlanguage}{brazilian}
    % \othertitle{Utilizando Word Embeddings para similaridade de palavras no Português Brasileiro}
\begin{abstract}
A capacidade de identificar a similaridade semântica entre palavras tem sido objeto de pesquisa nos últimos anos, pois está relacionada a uma série de atividades da área de processamento de linguagem natural, como recuperação de informação, sumarização de texto, categorização e geração, tradução automática e outros. A maioria dos sistemas de resposta a perguntas e extração de informações usa o WordNet para procurar sinônimos em seu mecanismo de busca. No entanto, a expansão de termos usando o WordNet tem vários problemas. Eles são de construção manual, demorados e caros, e por esse motivo, nem todos os links estarão presentes e sua qualidade varia de idioma para idioma, assim como não está disponível para todos os idiomas. Abordagens baseadas em distribuição, como a \textit{word embedding}, foram usadas para cobrir itens fora do vocabulário no WordNet. Assim, com a possibilidade de acesso a diferentes \textit{word embeddings models} e a necessidade de melhorar a forma de expandir os termos relacionados aos sistemas de consulta para bases ontológicas utilizadas por sistemas de perguntas e respostas e recuperação de informação, o presente trabalho explora as técnicas existentes para identificação de similaridade entre palavras, usando a abordagem distribuicional chamada word embeddings, adaptandando trabalhos existentes para o português brasileiro. Também é realizado experiementos com outras tecnicas que são basesadas em bases lexicas como WordNet, aonde uma avaliação qualitativa é realizada de todas as técnicas sobre um \textit{dataset} comun PT65. Provando que word embeddings podem de fato cobrir as palavras faltantes e tem um resultado ligeiramente melhor quando comparado com o WordNet. Também é realizado uma adaptação de estudos sobre a adição do contexto sintático no processo de trainamento do word embeddings a partir e um corpus português brasileiro, aonde obtivemos resultados similares atraves de uma avaliação qualitativa, porem para a atividade de identificar palavras similares utilizando o dataset PT65 os resultados foram piores em considersão aos outros modelos.
\palavraschave{Similaridade de palavras. WordNet. Word embedding. Linguística computacional. Processamento de Linguagem Natural.}
\end{abstract}
\end{otherlanguage}
