%=======================================================================
% Resumo em Português.
%
% A recomendação é para 150 a 500 palavras.
%=======================================================================
\begin{abstract}
The ability to identify the semantic similarity between words has been a subject of research explored in the last years, because it is an important support to a series of activities of the area of natural language processing like information retrieval, text summarization, categorization and generation, question answering, machine translation, and others.
Most of this systems that performs this tasks, use WordNet for synonym expansion, but often they face words out-of-vocabulary or have missing links between senses.
% The expansion of terms using WordNet has several problems. 
% They are of manual construction, time-consuming and expensive, and for this reason, not all links will be present and their quality varies from language to language, as also, it is not available for all languages. 
Distributional-based approaches like word embeddings have successfully been used to cover out-of-vocabulary items in WordNet. 
Thus, with the possibility of access to different word embeddings models and the need to improve the related terms expansion,  
% of query systems to ontological bases used by systems of questions answering and information retrieval
the present work explores the existing techniques regarding word similarity, using a distributional approach and adapting existing works to Brazilian Portuguese. 
We also experiment with the lexical database WordNet, and we do a qualitative evaluation of all the different techniques over a common dataset called PT65, indicating that word embeddings can cover words out of vocabulary and have better results in comparison with WordNet.
We also adapted the studies regarding the addition of syntactic context in the training process of word embeddings to a Brazilian Portuguese corpus, finding similar results through qualitative evaluation.
\palavraschave{Word similarity. WordNet. Word embedding. Computational linguistics. Natural Language Processing.}
\end{abstract}

%=======================================================================
% Resumo em língua estrangeira (sim, é aqui mesmo).
%
% O idioma usado aqui deve necessariamente aparecer nos parâmetros do
% \documentclass, no início do documento.
%=======================================================================
\begin{otherlanguage}{brazilian}
    % \othertitle{Usando Word Embedding para identificação de similaridade de palavras no Português Brasileiro}
\begin{abstract}
A capacidade de identificar a similaridade semântica entre palavras tem sido objeto de pesquisa nos últimos anos, pois oferece suporte a uma série de atividades da área de processamento de linguagem natural, como recuperação de informação, sumarização, categorização e geração de texto, tradução automática e outros.
A maioria dos sistemas que realizam estas atividades, usam WordNet para expansão de sinônimos, porém frequentemente eles não encontram alguns termos em seu vocabulário ou não possuem uma conexão entre seus \textit{synsets}.
Abordagens distribucionais, como a \textit{word embedding}, tem sido usada para cobrir termos fora do vocabulário no WordNet. 
Assim, com a possibilidade de acesso a diferentes \textit{word embeddings models} e a necessidade de melhorar a expansão de termos relacionados,
o presente trabalho explora as técnicas existentes para identificação de similaridade entre palavras, usando a abordagem distribucional, adaptando trabalhos existentes para o Português Brasileiro. 
Também é realizado experimentos com a base léxica WordNet, aonde uma avaliação qualitativa é realizada de todas as técnicas sobre o \textit{dataset} PT65, indicando que \textit{word embedding} pode de fato cobrir as palavras faltantes e tem um resultado melhor em comparação com o WordNet. Também é realizado uma adaptação de estudos sobre a adição do contexto sintático no processo de treinamento do \textit{word embedding} a partir e um corpus português brasileiro, onde obtivemos resultados similares através de uma avaliação qualitativa.
\palavraschave{Similaridade de palavras. WordNet. Word embedding. Linguística computacional. Processamento de Linguagem Natural.}
\end{abstract}
\end{otherlanguage}
