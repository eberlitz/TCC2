%=======================================================================
% Resumo em Português.
%
% A recomendação é para 150 a 500 palavras.
%=======================================================================
\begin{abstract}
The ability to identify the semantic similarity between words has been a subject of research very explored in the last years, because it is related to a series of activities of the area of natural language processing like information retrieval, text summarization, categorization and generation, database schema matching, question answering, machine translation, and others. Most of question answering and information extraction systems use WordNet to search for synonyms in their search engine. However, the expansion of terms using WordNet has several problems. They are of manual construction, time-consuming and expensive, and for this reason, not all links will be present and their quality varies from language to language, as also, it is not available for all languages. Distributional-based approaches like word embeddings have successfully been used to cover out-of-vocabulary items in WordNet. Thus, with the possibility of access to different word embeddings models and the need to improve the way of expanding related terms of query systems to ontological bases used by systems of questions answering and information retrieval, the present work explores the existing techniques regarding word similarity, using a distributional approach called word embeddings, adapting existing works to Brazilian Portuguese. We also experiment with other techniques that are solely based on a lexical database such as WordNet, and we do a qualitative evaluation of all the different techniques over a common dataset called PT65. Proving that word embeddings can cover words out of vocabulary and have slightly better results in comparison with WordNet. We also adapted the studies regarding the addition of syntactic context in the training process of word embeddings to a Brazilian Portuguese corpus, finding similar results through a qualitative evaluation, but for the task of word similarity against the dataset PT65, it had the worst results.
\palavraschave{Word similarity. WordNet. Word embedding. Computational linguistics. Natural Language Processing.}
\end{abstract}