% G
\subsection{Comparing Semantic Relatedness between Word Pairs in Portuguese Using Wikipedia}

In this paper, \citetexto{GranadaSV14} presents a new dataset for evaluating Distributional Similarity Models in Portuguese. For this, they translated the word pairs from the well-known baseline for semantic relatedness evaluation in English called RG65 created by \citetexto{Rubenstein1965ContextualCO}. The original dataset contains judgments from 51 human subjects for 65 word pairs. To generate the PT65 they translated all the word-pairs and evaluated them with 50 human subjects. They compared the human scores with previous works and also performed a qualitative evaluation using Latent semantic analysis (LSA) models generated from Wikipedia articles. The correlation scores obtained were close to the scores achieved by other works that targeted another language. With the experiment, they observed that the semantic similarity can be transferred across languages, but for Portuguese, a manual evaluation had better results. 



 

% Qual o problema abordado? hipotese?
% O que foi feito? Solução?
% Como foi realizada a avaliação?
% Comparar com o o meu trabalho - porque é relevante ao meu trabalho?
