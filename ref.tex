\section{Introdução}

Conforme \citetexto{Hexsel11}, a introdução tem o objetivo de ``\emph{introduzir} o material que vai ser apresentado em mais detalhe nas seções subseqüentes''. Na introdução você deve contextualizar o problema e mostrar por que vale a pena resolvê-lo. Você deve apresentar a solução proposta e mostrar o seu diferencial em relação aos trabalhos relacionados. Observe, porém, que na introdução você deve apenas tratar do O QUÊ e PORQUÊ, sem tratar do como \cite{Hexsel11}, que deve ser explicado na seção que descreve o trabalho desenvolvido.

Geralmente, a introdução tem uma estrutura similar ao resumo e deve apresentar:
\begin{itemize}
	\item \textbf{Contexto e motivação:} Aqui você deve apresentar o contexto do trabalho (área de que ele se trata) e uma motivação para trabalhar nesse assunto.
	\item \textbf{Problema:} Aqui você vai apresentar um problema, uma lacuna, observada na área e que você pretende tratar. Você deve se perguntar aqui: ``Que respostas estou disposto a responder?''. O problema deve ser definido claramente e delimitado em termos de espaço de tempo. Veja que essa parte visa alertar o leitor de que o que você está propondo é uma solução para um problema observado na área. 
	\item \textbf{Objetivos:} Aqui você deve apresentar os objetivos do seu trabalho. Tome cuidado para não confundir objetivos com atividades.   Faça a si mesmo a pergunta: ``O que pretendo alcançar com a pesquisa?''. Você pode discernir entre objetivos gerais e objetivos específicos:
	\begin{itemize}
		\item Objetivo geral --- qual o propósito da pesquisa?
		\item Objetivos específicos --- abertura do objetivo geral em outros menores (possíveis capítulos).
	\end{itemize}
	Veja abaixo um exemplo de objetivo retirado da monografia de~\citetexto{Teixeira09}:

	Com a possibilidade de acesso a base de dados XML gerada a partir do Sistema de Currículos Lattes e a necessidade de melhor reutilizar as informações existentes neste sistema, o presente trabalho tem como objetivo geral permitir o acesso do pesquisador a seus dados através de uma interface mais amigável: o padrão LaTeX. Para isto destacam-se os seguintes objetivos específicos:
	\begin{alineas}
		\item identificar e analisar o formato de especificação de currículos da Plataforma Lattes;
		\item disponibilizar uma ferramenta para a geração de uma representação de dados intermediária a partir do formato especificado;
		\item implementar a tradução dos dados colhidos em código LaTeX através da utilização da ferramenta criada;
		\item analisar os resultados obtidos e as alternativas presentes no uso da ferramenta.
	\end{alineas}
\end{itemize}




%=======================================================================
% Escrevendo o Texto
%=======================================================================
\section{Escrevendo o Texto}
Este capítulo apresenta algumas orientações para a escrita do texto.

\subsection{Comandos do \LaTeX}
Como regra geral, use os comandos tradicionais do \LaTeX\ para formatar seu texto.  Neste documento procuramos demonstrar os comandos mais comumente utilizados em monografias acadêmicas.

Neste capítulo apresentamos alguns exemplos de como colocar figuras e tabelas no seu texto.

A escrita de palavras estrangeiras deve ser feita através da macro ``foreign'', informando o nome do idioma como primeiro argumento. A utilização desta macro é importante para efetuar a hifenização correta, conforme idioma. Por exemplo, você pode escrever a palavra inglesa \foreign{english}{performance}.

\subsection{Ilustrações}
Aqui são apresentados alguns detalhes sobre o uso de ilustrações.

\subsubsection{Legendas}
As legendas das figuras devem se encontrar no topo da figura e não abaixo, como usualmente colocado. Abaixo da figura, é obrigatório colocar a fonte (mesmo que a figura tenha sido do próprio autor).

As legendas devem conter o tipo da ilustração (Figura, Tabela, etc), seguido de numeração simples (sem número do capítulo).

Toda figura deve ser citada no texto, como nos exemplos que seguem.

\subsubsection{Figuras}
A Figura~\ref{fig:escrita} ilustra as fases psicológicas da escrita da dissertação. Você vai se reconhecer no personagem. ;-)

\begin{figure}
	\caption{Fases psicológicas da escrita da dissertação}
	\label{fig:escrita}
	\centering%
	\begin{minipage}{.8\textwidth}
		\includegraphics[width=\textwidth]{escrita}
		\fonte{\citetexto{Cham12}}
	\end{minipage}
\end{figure}

\subsubsection{Tabelas e Quadros}
A Tabela~\ref{tab:estacoes} e o Quadro~\ref{tab:linguagens} são exemplos de tabela e quadro elaborados pelo(a) próprio(a) autor(a). Uma tabela apresenta informações onde há destaque de dados numéricos. Por outro lado, um quadro corresponde simplesmente a uma exibição de dados tabulados.

Além disso, quadros são formados por linhas horizontais e verticais, sendo considerados como elementos fechados. Já as tabelas são classificadas como abertas, pois não possuem linhas verticais.

\begin{table}
	\caption{Período das estações do ano no Brasil}
	\label{tab:estacoes}
	\centering%
	\begin{minipage}{.6\textwidth}
		\begin{tabular*}{\textwidth}{ll}
			\hline
			\textbf{Meses} & \textbf{Estações do Ano}\\
			\hline
			21 de março a 21 de junho & Outono\\
			21 de junho a 23 de setembro & Inverno\\
			23 de setembro a 21 de dezembro & Primavera\\
			21 de dezembro a 21 de março & Verão\\
			\hline
		\end{tabular*}
		\fonte{Elaborada pela autora.}
	\end{minipage}
\end{table}

\begin{board}
	\caption{Linguagens de Programação}
	\label{tab:linguagens}
	\centering%
	\begin{minipage}{.35\textwidth}
		\begin{tabular*}{\textwidth}{|l|l|}
			\hline
			\textbf{Nome} & \textbf{Criador}\\
			\hline
			C & Dennis Ritchie\\
			C++ & Bjarne Stroustrup\\
			Java & James Gosling\\
			PHP & Rasmus Lerdorf\\
			JavaScript & Brendan Eich\\
			\hline
		\end{tabular*}
		\fonte{Elaborada pela autora.}
	\end{minipage}
\end{board}




\subsection{Resumo}
O resumo deve conter de 150 a 250 palavras. No resumo não deve haver citações e indica-se que essa seja a última seção do texto a ser escrita. Veja abaixo uma sugestão de organização e exemplo de resumo de \citetexto{Moro11}.

Sugestão (uma a três linhas para cada item):
\begin{itemize}
	\item Contexto geral e específico;
	\item Questão/problema sendo investigado (propósito do trabalho);
	\item Estado-da-arte (por que precisa de uma solução nova/melhor);
	\item Solução (nome da proposta, metodologia básica sem detalhes, quais características respondem as questões iniciais, interpretação dos resultados, conclusões).
\end{itemize}

Exemplo (SANTOS et al., 2008 apud \citealp{Moro11}):
\begin{quote}
CONTEXTO: A Web é abundante em páginas que armazenam  dados de forma implícita. PROBLEMA: Em muitos casos, estes dados estão presentes em textos semiestruturados sem a presença de delimitadores explícitos e organizados em uma estrutura também implícita. SOLUÇÃO: Este artigo apresenta uma nova abordagem para extração em textos semi-estruturados baseada em Modelos de Markov Ocultos (Hidden Markov Models - HMM). ESTADO-DA-ARTE e MÉTODO PROPOSTO: Ao contrário de outros trabalhos baseados em HMM, a abordagem proposta dá ênfase à extração de metadados, além dos dados propriamente ditos. Esta abordagem consiste no uso de uma estrutura aninhada de HMMs, onde um HMM principal identifica os atributos no texto e HMMs internos, um para cada atributo, identificam os dados e metadados. Os HMMs são gerados a partir de um treinamento com uma fração de amostras da base a ser extraída. RESULTADOS: Os experimentos realizados com anúncios de classificados retirados da Web mostram que o processo de extração alcança qualidade acima de 0,97 com a medida F, mesmo se esta fração de treinamento é pequena. 
\end{quote}

%=======================================================================
% Exemplos de Citações e Referências Bibliográficas
%=======================================================================
\section{Exemplos de Citações e Referências Bibliográficas}
\nobibliography* % para usar o \bibentry
Neste capítulo são apresentados exemplos de citações e referências bibliográficas.  Aqui é utilizado o pacote \texttt{bibentry}, que permite a inserção de referências no meio do texto (atenção para a diferença entre citações e referências).

Você vai ver que, neste exemplo, não está sendo usado o estilo de referências bibliográficas do projeto ABNTeX\footnote{http://http://sourceforge.net/projects/abntex}.  Você é completamente livre para usá-lo (veja no início do arquivo .tex como fazer isso).  Os motivos para não usar o ABNTeX neste exemplo são basicamente dois:
\begin{itemize}
	\item Para usar o ABNTeX, é necessário instalá-lo em seu sistema \TeX\ primeiro; embora não seja uma tarefa tão complicada, enxergamos como uma dificuldade a mais para o usuário iniciante.  Nosso objetivo aqui é facilitar ao aluno da UNISINOS o uso deste modelo, de modo que basta copiar os arquivos \texttt{UNISINOSmonografia.cls} e \texttt{unisinos.bst} para a pasta onde estão seus arquivos .tex;
	\item As normas da ABNT são tão complexas que, para atender a todas as variações possíveis de citações e referências, o projeto ABNTeX criou uma série de campos adicionais nas entradas do arquivo .bib.  Embora funcione para o caso ABNT, o efeito colateral de fazer isso é que o seu arquivo .bib será muitas vezes incompatível com os demais estilos tradicionais do BibTeX, como \texttt{plain}, \texttt{alpha}, \texttt{ieeetr}, entre outros.  Por exemplo, em referências a artigos publicados em conferências, o campo \texttt{organization} é usado pelo ABNTeX para definir o nome do evento.  Isso não é padrão e não será reconhecido pelos estilos tradicionais\footnote{Veja como criar seus arquivos .bib no manual do BibTeX, que pode ser encontrado em http://ctan.tug.org/tex-archive/biblio/bibtex/contrib/doc/btxdoc.pdf.}.  Considerando que um dos maiores benefícios do BibTeX é criar um arquivo .bib que pode ser reutilizado pelo resto da vida, nossa estratégia com o \texttt{unisinos.bst} foi tentar aproximar ao máximo a formatação exigida pela ABNT sem implicar na criação de arquivos .bib incompatíveis.  Isso funciona bem na grande maioria dos casos, mas não em todos.  Nesse caso, a saída é usar o ABNTeX ou então alterar manualmente o arquivo .bbl que é gerado ao rodar o comando \texttt{bibtex}.
\end{itemize}

Em caso de dúvida, siga as orientações do guia da Biblioteca \cite{Biblioteca12} e, se necessário, da norma NBR~6023 \cite{NBR6023:2002}.

\subsection{Citações}
As citações podem ocorrer de duas formas: com os nomes dos autores inseridos no texto ou não.  Isso implica em uma construção diferente para as frases.  Por exemplo:
\begin{itemize}
	\item Com o nome do autor inserido no texto: ``De acordo com \citetexto{Tanenbaum03}, o modelo de referência OSI foi proposto de forma tardia.''
	\item Sem inserir o autor no texto: ``O modelo de referência OSI foi proposto de forma tardia \cite{Tanenbaum03}.''
\end{itemize}

\subsection{Livros}
Seguem alguns exemplos de referências de livros:
\begin{itemize}
	\item \bibentry{Buford09}.
	\item Livro com indicação de edição:\\
	\bibentry{Kurose10ptbr}.
\end{itemize}

\subsection{Artigos em Periódicos}
Os exemplos abaixo ilustram referências a artigos em periódicos.
\begin{itemize}
	\item \bibentry{Hayes08}.
	\item \bibentry{Lawton08}.
\end{itemize}

\subsection{Artigos em Conferências}
\begin{itemize}
	\item \bibentry{Laadan10}.
	\item \bibentry{Anderson95}.
\end{itemize}

\subsection{Teses e Dissertações}
Seguem algumas referências a trabalhos acadêmicos, como teses, dissertações, trabalhos de conclusão de curso, etc.
\begin{itemize}
	\item \bibentry{Teixeira09}.
	\item \bibentry{Flaumann05}.
\end{itemize}

%=======================================================================
% Conclusão
%=======================================================================
\section{Conclusão}
Não se esqueça de terminar o artigo com uma conclusão. :-)
