\section{Final considerations}\label{chap:conclusions}

The study carried out in this work, proves that the detection of similarity between words is a very important topic for the NLP segments as summarization, information retrieval, and question answering. More precisely techniques for identifying word similarity can help in a number of NLP tasks such as dialogue systems, question answering, and information retrieval systems. \cite{Islam2007ApplicationsOC,Pilehvar2013,Agirre2009}.

It was also found indications that the expansion of terms using WordNet has several problems, where a word may not be present. Since these lexical bases are of manual construction, they are time-consuming and expensive, and for this reason, not all links will be present and their quality varies from language to language.  There is also no WordNet for all languages. \cite{Leeuwenberga2016}. Distributional approaches regarding word similarity have been proven to be more competitive than the thesaurus-based approach, and have been successfully being used to cover out-of-vocabulary items in WordNet.  \cite{Agirre2009}.

% e que cada vez mais está sendo utilizado word embeddings para downstream tasks. Também tem-se uma certa tendência de estudos na área misturando arvores sintaticas no treinamento. \cite{Cotterell2015MorphologicalW, Levy2014}. 

Thus, WordNet is proposed to be replaced by Word embeddings, which follows a distributional approach and therefore does not depend on manual construction, and can be applied to different languages since its training is unsupervised. Thus, the hypothesis is that for the formulation of queries in QA and IR systems on which they depend on the expansion of similar terms it would be possible to increase the number of relevant results to be found.






 
 